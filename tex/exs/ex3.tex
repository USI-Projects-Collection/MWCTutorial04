\section{Exercise 3 – Step Counter}

\subsection{Android STEP\_DETECTOR}

First of all I modified the variable \texttt{accSensor} in the \texttt{StepsFragment} class to be of type:\\ \texttt{Sensor.TYPE\_STEP\_DETECTOR}. I could have easly have defined another variable but I wanted to make sure that all the counts were updted thanks to this sensor and not the previsouly used\\ \texttt{Sensor.TYP\_LINEAR\_ACCELERATION}.

\begin{verbatim}
    accSensor = sensorManager.getDefaultSensor(Sensor.TYPE_STEP_DETECTOR);
\end{verbatim}

Then I moved to the \texttt{StepCounterListener} class and mofied the \texttt{onSensorChanged} method to call the \texttt{countSteps} method. 

\lstinputlisting[
    firstline=101, 
    lastline=103, 
    firstnumber=101,
    caption=\texttt{\codepath/StepCounterListener.java}
]{\codepath/StepCounterListener.java}

\lstinputlisting[
    firstline=150, 
    lastline=161, 
    firstnumber=150,
    caption=\texttt{\codepath/StepCounterListener.java}
]{\codepath/StepCounterListener.java}

This method is used to update the step count, the step count text view and the step count progress bar (for the )